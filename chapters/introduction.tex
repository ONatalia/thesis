\chapter{Introduction}
\label{chap:introduction}
\section {Motivation}
In recent years, Automatic Speech Recognition (ASR) systems have improved
increasingly, being used in everyday applications: Siri, Google ASR 
\parencite {mcgrawgrauenstein2012}. However, most of such systems work
asynchronously in respect to output, computing the result after the utterance is already finished.
In the area of Human-Machine Interaction, 
where intermediate system reactions of ASRs are 
desirable,  incremental output of the intermediate results becomes increasingly
important. Benefits of  incremental speech recognition include post-processing
time savings, faster system feedback and more natural dialogue between humans
and intelligent systems. Commercial Google recognition
engine, working in incremental mode, produces accurate
results in non-specific domains, but demonstrates higher latency
\parencite{mcgrawgrauenstein2012}.  Non-commercial open source systems, like Sphinx-4, on the other hand, demonstrate very short delays and can be timed to specific
applications, but are less reliable in accuracy.
The challenge is to combine the advantages of both systems and  to overcome the latency problem of Google-ASR.
\section {Problem Statement}
The aim of this master thesis is to investigate the possibility of
timeliness and timing improvement of Google ASR
incremental results by developing an incremental speech recognizer, using a
combination of Google ASR \parencite
{mcgrawgrauenstein2012} and a Sphinx-4 speech recogniser
\parencite {baumannetal2009:naacl}. 
\section {Objectives}

The following objectives states, how the aim of the master thesis is going to be
addressed:
 \begin  {itemize}
   \item to forced-align incremental results,coming from Google ASR, using a
   combination of Google and Sphinx ASRs
   \item to evaluate timing results of forced-alignment, comparing Google
   alone, Sphinx alone and Google + Sphinx combination, using evaluation metrics
   \item to investigate the possibility for Google timeliness improvement in multiple source recogniser
   \item to evaluate results of timeliness improvement, comparing Google alone,
   Sphinx alone and a Google + Sphinx combination
 \end {itemize}


\section {Structure}
This master thesis is structured as follows. Chapter \ref{ch:relatedWork}
provides review of state of the art scientific papers, related to the thesis
topic. Chapter \ref{chap:terms} explains the terminology to be used in the
context of this master thesis.  Chapter \ref{chap:sphinx} describes the
architecture of CMU (Carnegie Mellon University) Sphinx-4 recognizer.
Implementation of a Google+Sphinx recognizer is discussed in the chapter \ref{chap:implem} of this paper. Chapter \ref{chap:res} describes the
evaluation approach and results. Chapter \ref{chap:concl} summarizes the master
thesis.
