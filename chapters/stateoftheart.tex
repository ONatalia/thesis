\chapter {Related Work}
\label{ch:relatedWork}
The architecture of  contemporary speech recognition systems includes basic ASR
components: acoustic input, acoustic model, language model and decoder.
ASR decoder gets as input feature vectors of audio signal and the results
of acoustic and language modelling and produces a decoded word and
phone alignment as an output \parencite {jurafskymartin2009}.

Most traditional ASRs, including commercial, are not incremental. They produce
the output result after the utterance is finished, resulting in a delay up to
750-1500 ms. In additional they may use multiple-pass decoding, which is not
possible incrementally \parencite {skantzeschlangen2009}.
Incremental dialogue processing allows reducing of feedback time, bringing the
dialogues in interactive environments closer to natural ones.
Apart from abstract models of incremental dialogue processing there exist an
implementation  of a limited micro-domain system, recognising sequence
numbers \parencite {skantzeschlangen2009}.

A good example of  traditional ASR architecture is the implementation of the
open-source CMU Sphinx-4 recogniser \parencite {Lamereetal2013:Eurospeech}.
Sphinx-4 was not designed as an incremental system, but its modular design 
allows extension and adding new components, required for an incremental speech recogniser. 
\parencite {baumannetal2009:naacl}.

Apart from Sphinx-4 the leading speech recognition systems is commercial
Google Voice Search \parencite{schalkwyk2010}.

State-of-art Google Search by Voice, trained using huge amounts of audio data,
presents the best results in the terms of accuracy for a
standard non-incremental task. However, when intermediate results are produced
synchronously in an incremental mode we see a trade-off between stability of the
input and latency \parencite {mcgrawgrauenstein2012}. 

Further restriction of Google is its task-orientated recognition, primarily 
aimed to interactive Google command search. This generally means that simple
transfer of Google technology to domain-specific natural dialogue systems and
context-specific HRI environments leads to lower accuracy. 

Recently, there has been proposed a solution for a domain-dependent
applications, using a combination of Google phonetic post processing and
Sphinx-4. Original frontend of Sphinx-4 is replaced by phoneme frontend, which
converts the Google result to its phonetic representation. Test
results of the combined approach have shown a significant improvement of
recognition results for non-incremental tasks.
Combination technique is considered to be transferable to the
incremental problem solution \parencite {twiefeletal2014}.
