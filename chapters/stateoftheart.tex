\chapter {Related Work}
\label{ch:relatedWork}
Although architecture of  all contemporary speech recognition systems includes
such common ASR components as acoustic input, acoustic model, language model,
search module and decoder, significant differences in performance and 
application approaches are still to be observed due to the system complexity. 


During the process of speech recognition decoder gets feature vectors of audio
signal and results of acoustic and language modelling as input and produces
a decoded word-phone alignment as an output \parencite {jurafskymartin2009}.
Traditionally ASRs were designed sequentially, producing the first output result
after the utterance is already finished, thus resulting in a delay up to
750-1500 ms. By contrast, incremental speech recognition, which can be
seen as a part of dialogue processing systems, allows reducing of feedback time,
bringing the dialogues in interactive
environments closer to natural ones \parencite {skantzeschlangen2009}. 
Such dialogues systems can be understood very widely, including verbal and/or
non-verbal communication. In all cases reactivity, meaning a prompt
incremental result of the speech recognizer, is seen as important criteria for
the system performance. 

A good example of  traditional ASR architecture is the implementation of the
open-source CMU Sphinx-4 recogniser. Sphinx-4 is an open-source local
recognition system, that allows tuning for specific applications and embedded
environments \parencite
{Lamereetal2013:Eurospeech}.  Word Error Rate (WER) of Sphinx-4 varies from 0.168 for isolated
digit recognition to 18.7 for large vocabulary (Carnegie Melon University).
Though, Sphinx-4 was not designed as an incremental system, its modular design allows extension and adding new components, required 
for an incremental speech recogniser. As a result of such extension an
incremental speech recognizer, based on sphinx-4 was implemented. The authors present very detailed analyses of timing for
incremental results of Sphinx ASR \parencite {baumannetal2009:naacl}.

Apart from Sphinx-4 the leading speech recognition systems is commercial
Google Voice Search \parencite{schalkwyk2010}. Google runs in non-incremental as
well as incremental mode. In comparison to Sphinx-4  cloud-based Google ASR uses
language models, trained with  large amount of audio data, presents more
accurate results in the terms of WER. At the present moment WER of Google ASR for general tasks
with large vocabulary equals about 8\%. However, there is always trade-off
between stability of the input and latency \parencite {mcgrawgrauenstein2012}.

Further restriction of Google is its task-orientated recognition, primarily 
aimed to interactive Google command search. This generally means that simple
transfer of Google technology to domain-specific natural dialogue systems and
context-specific HRI environments leads to lower accuracy. Besides, as Google
ASR recognizer features, such as, for example, timing information, that present
research interest in domain-dependent and/or embedded application, remains encapsulated in a ``black box''. 

Recently, there has been proposed a solution for a domain-dependent
applications, using a combination of Google phonetic post processing and
Sphinx-4. Original frontend of Sphinx-4 is replaced by phoneme frontend, which
converts the Google result to its phonetic representation. Test
results of the combined approach have shown a significant improvement of
recognition results for non-incremental tasks. Combination technique is
considered to be transferable to the incremental problem solution \parencite {twiefeletal2014}.





