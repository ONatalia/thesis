\chapter{Correspondence between Sphinx phoneme notation and German phonemes }
\label{chap:appA}
\begin {longtable}{| c | c |}
%\label{tab:alignment_incr} 
%\begin{center}
    %\begin{tabular}{| c | c |}
    \hline
    Sphinx dictionary notation & German phoneme \\ \hline
    2:  &  \\ \hline
    6 &  \\ \hline
    9  &  \\ \hline
    C  &  \\ \hline
    E  & \\ \hline
    E:  & \\ \hline 
    I  & \\ \hline 
    N  & \\ \hline 
    O  & \\ \hline 
    OY  & \\ \hline 
    Q  & \\ \hline 
    S  & \\ \hline 
    U  & \\ \hline 
    Y  & \\ \hline 
    Z  & \\ \hline 
    a  & \\ \hline 
    a:  & \\ \hline
    aI   & \\ \hline
    aU  & \\ \hline
    b  & \\ \hline
    d  & \\ \hline
    e  & \\ \hline
    e:  & \\ \hline
    f  & \\ \hline
    g  & \\ \hline
    h  & \\ \hline
    i  & \\ \hline 
    i  & \\ \hline 
    i:  & \\ \hline 
    j  & \\ \hline 
    k  & \\ \hline 
    l  & \\ \hline 
    m & \\ \hline 
    n  & \\ \hline 
    o  & \\ \hline 
    o:  & \\ \hline 
    p  & \\ \hline 
    r  & \\ \hline 
    s  & \\ \hline 
    t  & \\ \hline 
    u  & \\ \hline 
    u:  & \\ \hline
    v  & \\ \hline
    x  & \\ \hline 
    y:  & \\ \hline 
    z  & \\ \hline   
    %\end{tabular}

%\end{center}
\end {longtable}

 