\chapter{Results and Evaluation}
\label{chap:res}
 \section {Testing Approach} 
For testing purposes  a corpus of  Verbmobil-Data (dialogues about meeting
appointments)  is available.  Each record contains audio file, used by the recognizer as input in
one of the configurations,  and \textit {gold standard}
with the true words from the audio, including true alignments.  Depending on the program configuration 
 the following types of outputs were compared:
\begin {enumerate}
  \item Google incremental output (1).
  \item Sphinx incremental output: Sphinx in recognition mode, using available
  language model (2).
  \item Google+Sphinx: Sphinx in forced alignment mode after Google
  incrementally, alignment of the incremental results as described in the
  chapter \ref {chap:implem} (3).
  \item Google+Sphinx: Sphinx in forced alignment and recognition mode,
  combined alignment and phone loop grammar (4).
  \item Google+Sphinx: Sphinx in forced alignment and recognition mode,combined
  alignment grammar and the same language model as in item number 1 (5). 
\end{enumerate}
\section {Test Results} 
\subsection {Alignment quality}
\begin {table}
\label{tab:alignment_non_increm} 
\begin{center}
\caption {Alignments quality of the recognizer in different configurations.}
    \begin{tabular}{l  c  c  c }
   \toprule
    Recognizer & Mean & stddev & RMSE \\ \toprule
    Google (1)  & ms & ms & ms \\ 
    Sphinx (2)  & ms & ms & ms \\ 
    Google+Sphinx (3)  & ms & ms & ms \\ 
    Google+Sphinx (4)  & ms & ms & ms \\ 
    Google+Sphinx (5)  & ms & ms & ms \\ \bottomrule  
    \end{tabular}
\end{center}
\end {table}
\begin {table}
\label{tab:alignment_incr} 
\begin{center}
\caption {Incremental alignments quality of the recognizer in different
configurations.}
    \begin{tabular}{ l  c  c  c }
    \toprule
    Recognizer & mean  & stddev & RMSE\\ \toprule
    Google (1)  & ms & ms & ms \\ 
    Sphinx (2)  & ms & ms & ms \\ 
    Google+Sphinx (3)  & ms & ms & ms \\ 
    Google+Sphinx (4)  & ms & ms & ms \\ 
    Google+Sphinx (5)  & ms & ms & ms \\ \bottomrule  
    \end{tabular}
\end{center}
\end {table}
\subsection {Timeliness quality}
\begin {table}
\label{tab:alignment_incr} 
\begin{center}
\caption {Timeliness quality of the recognizer in different configurations. FO
results.}
    \begin{tabular}{ l  c  c  c }
    \toprule
    Recognizer & mean & stddev & median \\ \toprule
    Google (1)  & 3333 ms & 3715 ms &  1480 ms\\
    Sphinx (2)  &284 ms & 709 ms & 130 ms \\
    Google+Sphinx (3)  & ms & ms & ms \\ 
    Google+Sphinx (4)  & ms & & \\ 
    Google+Sphinx (5)  & ms & &  \\ \bottomrule  
    \end{tabular}
\end{center}
\end {table}

\begin {table}
\label{tab:alignment_incr} 
\begin{center}
\caption {Timeliness quality of the recognizer in different configurations. FD
results.}
    \begin{tabular}{ l  c  c  c }
    %\hline
    \toprule
    Recognizer & mean & stddev & median \\ \toprule
    Google (1)  & 3833 ms & 4271 ms & 1725 ms  \\ 
    Sphinx (2)  & 324 ms & 705 ms & 170 ms \\ 
    Google+Sphinx (3)  & ms & ms & ms \\ 
    Google+Sphinx (4)  & ms  & & \\ 
    Google+Sphinx (5)  & ms  & &\\ \bottomrule   
    \end{tabular}
\end{center}
\end {table}

\section {Tests Evaluation} 