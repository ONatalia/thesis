\chapter*{Abstract}
In recent years, Automatic Speech Recognition (ASR) systems have improved
increasingly, being used in everyday applications: Siri, Google ASR However,
often such systems work asynchronously in respect to output, computing the
result after the utterance is already finished.
In the area of Human-Machine Interaction, 
where intermediate system reactions of ASRs are 
desirable,  incremental output of the intermediate results becomes increasingly
important. Benefits of  incremental speech recognition include post-processing
time savings, faster system feedback and more natural dialogue between humans
and intelligent systems. Commercial Google recognition
engine, working in incremental mode, produces accurate
results in non-specific domains, but demonstrates higher latency
Non-commercial open source systems, like Sphinx-4, on the other hand, demonstrate very short delays and can be timed to specific
applications, but are less reliable in accuracy.
The challenge is to combine the advantages of both systems and  to overcome the latency problem of Google-ASR.
% \section {Problem Statement}
% The aim of this master thesis is to investigate the possibility of
% timeliness and timing improvement of Google ASR
% incremental results by developing an incremental speech recognizer, using a
% combination of Google ASR \parencite
% {mcgrawgrauenstein2012} and a Sphinx-4 speech recogniser

\begin{otherlanguage}{ngerman}
\section*{Zusammenfassung}
\end{otherlanguage}