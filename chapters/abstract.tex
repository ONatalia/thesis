\chapter*{Abstract}
% In recent years, Automatic Speech Recognition (ASR) systems have improved
% increasingly. However, often such systems work non-incrementally, producing 
% the end result after the utterance is already finished. For applications, where intermediate system reactions of ASRs are desirable,  
% incremental output of the intermediate results becomes increasingly important. Benefits of  incremental speech recognition 
% include post-processing time savings, faster system feedback and more natural dialogue between humans and intelligent systems. 
% Commercial Google recognition engine, able to run in incremental mode, produces accurate results in non-specific domains,
%  but demonstrates higher latency. In addition Google results provide no  timing information for the words, occurring in  the recognized sentences. 
%  Forced alignment, absent in Google, is of great importance for dialogue systems, involving non-verbal recognition.  
% Non-commercial open source Sphinx-4 system, on the other hand, demonstrates very short delays and can be timed to specific applications, but is 
% less reliable in accuracy.
This master thesis investigates the possibilities of Google+Sphinx combination within the frames of Incremental Spoken Dialogue 
Processing Toolkit (InproTK). By contrast to Google the combination of
Google+Sphinx returns timing information, which is of great importance to
dialogue systems, involving non-verbal recognition. The central part of the the
proposed architecture is a Incremental Unit (IU) module, which guarantees that Google incremental results are passed to 
Sphinx recognizer, working in a forced-alignment mode. Google+Sphinx alignment results as well as the results of
Google and Sphinx alone are evaluated against the gold standard, using
timing metrics. Furthermore, we propose approaches, allowing combination of
alignment and recognition within  Google+Sphinx architecture. The second aspect
of the analyses is the Sphinx-4 search graph path alternations strategies.
Finally, we discuss the problems of timing and timeliness results evaluation.

% In recent years, Automatic Speech Recognition (ASR) systems have improved
% increasingly, being used in everyday applications: Siri, Google ASR However,
% often such systems work asynchronously in respect to output, computing the
% result after the utterance is already finished.
% In the area of Human-Machine Interaction, 
% where intermediate system reactions of ASRs are 
% desirable,  incremental output of the intermediate results becomes increasingly
% important. Benefits of  incremental speech recognition include post-processing
% time savings, faster system feedback and more natural dialogue between humans
% and intelligent systems. Commercial Google recognition
% engine, working in incremental mode, produces accurate
% results in non-specific domains, but demonstrates higher latency
% Non-commercial open source systems, like Sphinx-4, on the other hand, demonstrate very short delays and can be timed to specific
% applications, but are less reliable in accuracy.
% The challenge is to combine the advantages of both systems and  to overcome the latency problem of Google-ASR.
% \section {Problem Statement}
% The aim of this master thesis is to investigate the possibility of
% timeliness and timing improvement of Google ASR
% incremental results by developing an incremental speech recognizer, using a
% combination of Google ASR \parencite
% {mcgrawgrauenstein2012} and a Sphinx-4 speech recogniser

\begin{otherlanguage}{ngerman}
\section*{Zusammenfassung}
Diese Masterarbeit untersucht die Möglichkeiten Google+Sphinx im Rahmen von Incremental Spoken 
Dialogue Processing Toolkit (InproTK) zu kombinieren. Im Vergleich zu Google die Kombination Google+Sphinx 
liefert  zeitliche Information, die sehr wichtig für Dialogsystemen mit
non-verbal Erkennung ist.  Das zentralen Teil von vorgestelltem Architektur ist 
Incremental Unit (IU) Modul, das garantiert, dass die incrementelle Google
Ergebnisse werden an Sphinx Erkenner, gestartet  im forced alignenment Modus,
weiter geleitet. Ergebnisse sowie von Google+Sphinx, als auch von Google und
Sphinx allein werden gegen gold Standard evaluiert, dabei werden zeitliche
Metriken benutzt. Weiterhin, we schlagen vor die Vorgehensweise, um die Kombination von Alignment und Erkennung in Rahmen von Google+Sphinx Architektur zu gewährleisten. Der zweiten Aspekt von Analyse sind
die Strategien, die erlauben der Pfad im Sphinx-4 Suchgraph zu ändern. Zum
Schluss werden die Problemen von zeitlichen Ergebnissen besprochen.
\end{otherlanguage}