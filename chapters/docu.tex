\documentclass[a4,twoside=off,draft=true]{scrartcl}
\usepackage[utf8]{inputenc}
\usepackage[pdftitle={(IN)TELIDA Documentation},
            pdfauthor={Timo Baumann},
            linktocpage,
           ]{hyperref}
\usepackage{graphicx}
\usepackage{tipa} % needed for \textpipe
\usepackage{textcomp} % needed for \textlangle \textrangle

\begin{document}

%INcremental TImed LInguistic DAta
\title{(IN)TELIDA Programmer's Manual}
\subject{Inpro Technical Document 2009/03-893X3Ma2}
\author{Inpro, Timo Baumann}
\date{Version of \today}

\maketitle
\tableofcontents
\clearpage

\section{Introduction}

We describe (IN)TELIDA, a library, toolkit, toolbox or what-you-name-it 
to work with (\emph{in}cremental) \emph{t}im\emph{e}d \emph{li}nguistic \emph{da}ta.

\subsection{Typographic Conventions}

Method and variable names should be rendered like {\tt this}. 

Method signatures are given like the following: 
{\tt Util::min(\$a, \$b) :\ \$}, with the method name preceded by the module name,
parameters given in parentheses ({\tt \$a} and {\tt \$b} in this case) and a return
value after a colon. Perl is a loosely typed language and requires the programmer
to think before they type, as most operations will still succeed (but with unexpected
results) if preconditions are not met. The return types of functions are especially
prone to this type of problem.
Whenever possible we try to be as precise as possible about 
the requested and returned values. For example, if a parameter should be a reference
to a list, we write {\tt \textbackslash@list1} instead of just {\tt \$list1} and
enumerate returned values in a list if their meaning is not obvious.

\section{Data Structures}

We handle timed data as discrete \emph{labels} which span a certain time and contain some 
data.\footnote{
 Note that it is hard to model gradual changes of continuous variables in this way. 
}
Labels which contain information of the same type (e.\,g.\ about what word was said at what time)
are collected into a sequence of labels, which we call alignment. 
An \emph{alignment} documents what we think (with regard to information of some kind) about a certain
time span and maybe at a certain time. We capture what we think about the individual times
using labels. Labels in an alignment must not overlap. 

It is often desirable or even necessary to have collections of alignments which contain
information of different types (words, phonemes, turns, \ldots) that all describe the same
time span. For a given audio file we might be interested in knowing what turns were said
and how the turns decompose into words and phonemes. If we have dual-channel audio with separate
speakers on each channel, we may have different alignments for each speaker. And in a multi-modal
setting we may have non-auditory information (such as mouse gestures, visual feedback, \ldots) 
in separate alignments.
The original format and annotation tool we used for this is Praat and its TextGrid format,
so we call collections of alignments \emph{TextGrid}.

TextGrids work fine for collections of timed data for the same time span but
they are unsuitable to process and store incremental timed linguistic data (that is, data which
becomes available at certain points in time and usually describes the alignment of labels
up to this point in time -- or even up into the future). 
A TextGrid is made to store aligned data of different types but all alignments are 
describing (roughly) the same  time span.
In contrast to this, an \emph{AlignmentSequence} contains multiple alignments of the same type
which only differ with regard to the time they were created (or the time at which they were valid).
An alignment sequence allows us to capture our thinking about what happens as time proceeds. 
To capture our changes of opinion about what happens, we use \emph{EditMessages} that describe
the differences between adjacent alignments.
For an example of an alignment sequence and edit messages see Figure~\ref{fig:reco_example}.

\begin{figure}
\includegraphics[scale=1.0]{incremental_reco}
% figure exported from incremental_reco.odg
\caption{\label{fig:reco_example}An \emph{AlignmentSequence} showing \emph{live} ASR hypotheses during incremental recognition.
Edit messages are shown on the right 
when words are added ($\oplus$) or revoked ($\ominus$).
For the word ``{\small\textsf{zwei}}'' two timing measures
are shown at the bottom.}
\end{figure}

We may not always know about the right alignment or we may have multiple opinions about
what alignment might be appropriate. This typically applies to speech recognition systems while
they are processing. They follow a bunch of hypotheses about the alignment of words and only later 
(as more information becomes available) decide on which alignment to pick. To capture this case, 
we define an \emph{NBestList} to contain one or more alignments that record what we consider
at a certain point in time. To study temporal aspects of n-best list changes,
we use the \emph{NBestAlignmentSequence} which stores a consecutive collection of n-best lists.

\section{Modules}

\subsection{Timed Data}

This subsection covers data structures and methods suitable for non-incremental timed data, which
we cover from bottom (i.\,e.\ smaller structures) to top (i.\,e.\ larger containing structures).

\subsubsection{Label.pm}

Objects of the {\tt Label} class contain information of what happens in a given time span.
Thus, they generally contain the properties {\tt xmin} and {\tt xmax} for the time span 
and {\tt text} for the information contained. The optional property {\tt color} encodes
the color of the label when displayed in TEDview.

It must always hold that ${\tt xmin} \leq {\tt xmax}$
and {\tt text} must not contain a newline character (\textbackslash{n}).

Likely extension: Labels might become typed, that is might know that they can only contain information of 
a certain kind (for example only matching the regexp {\tt /task\textpipe{}non-task\textpipe{}non-sys\textpipe{}/}).
Also it would be nice if labels knew their predecessors and successors or could hold links to temporally
related labels from other tiers (like words knowing their phonemes and vice-versa, or pronouns
knowing their referents).

Labels support the following operations:

\begin{itemize}
\item {\tt new(\$xmin, \$xmax, \$text) :\ \textbackslash{\%Label}} \\
 Usually Labels are created for you automatically when you load a TextGrid or Alignment file. 
 If you do need to create a label, then supply xmin, xmax and text as arguments.
\item IO-functions to read labels from various formats and write them accordingly. 
 If you need to call one, it's likely {\tt toWavesurferLine()} which does not 
 take any arguments and returns the fields in a tab-separated line.
\item {\tt isEmpty()} checks whether the label is blank
\item {\tt isSilent()} checks whether the label is blank or contains only a silence marker
\item {\tt equals(\$other, {\$mode})} \\
 checks whether the label equals the other label. There are two modes
 for equality: if \$mode is 'lax', only label texts must be identical, 
 if \$mode is 'strict' (which is the default), then times are compared as well.
\item {\tt duration() :\ \$} \\
 returns the duration of the label (that is, the difference between xmax and xmin)
\item {\tt durationInDots(\$dotDuration, \$mode) :\ \$} \\
 returns a string representation that shows the duration of the label in dots. 
 This is especially useful to signify the duration of pauses in text-only 
 output ("Nimm das .... linke Teil."), or could again be used in text-only output 
 to visualize the duration of any sequence of labels in alignment collections. 
 The duration of a dot in seconds is given as first argument (a dot is returned for 
 the start of every such interval). Again, there are two modes: if \$mode is 'log', 
 then the duration of the labels is first logarithmized (actually,
 $log(1 + duration)$) so that long labels do not become ridiculously dotty.
\item {\tt getTimings(\$time)} \\
 given a time, returns the absolute differences between this label's start-time and end-time 
 and a relative position of the given time normalized by the word's duration
\item {\tt timeShift(\$time)} \\
 shifts the label start and end point by the time given as argument
\item {\tt setColor(\$color)} \\ 
 sets the color of the label that will be used when displaying/exporting to TEDview.
 {\tt \$color} may contain arbitrary text but \emph{should} only contain color values
 interpretable by TEDview, i.\,e.\ some named colors ({\tt white}, {\tt red}, \ldots) as
 well as HTML color codes ({\tt \#123456}).
\end{itemize}

\subsubsection{Alignment.pm}
An {\tt Alignment} object is a collection of {\tt Label} objects, represented as a list property.
An object of this kind also contains the features {\tt xmin} and {\tt xmax}, which represent its time span. 
These values can be set automatically with regard to the overall time spanned by the contained alignments or entered manually - in every case the boundaries {\tt xmin} and {\tt xmax} of each label have to be within the overall boundaries. 

The following operations can be used concerning an {\tt Alignment} object:

\begin{itemize}

\item {\tt newEmptyAlignment() : \textbackslash{\%Alignment}} 
\item {\tt checkLabels()} 
\item {\tt clone(\$time)} 
\item {\tt newAlignmentFromTextGridLines(\textbackslash@lines)} 
\item {\tt toTextGridLines() : \textbackslash@lines } 
\item {\tt newAlignmentFromWavesurferLines(\textbackslash@lines)} 
\item {\tt newAlignmentFromWavesurferFile(\$filename)} 
\item {\tt toWavesurferLines () : \textbackslash@lines } 
\item {\tt saveToWavesurferFile(\$filename)} 
\item {\tt newAlignmentFromSphinxLines (\\textbackslash@lines)} 
\item {\tt toSphinxLines() : \textbackslash@lines} 
\item {\tt toTEDXMLLines(\$time, \$orig, \$diamond) : \textbackslash@lines } 
\item {\tt getName() : \$name} 
\item {\tt setName(\$name)} 
\item {\tt getLabels(\$time)} 
\item {\tt setLabels(\$time)} 
\item {\tt getTime() : \$time } 
\item {\tt getLabelAt(\$time)}
\item {\tt getWords(\$time)} 
\item {\tt getWordsUpTo(\$time)} 
\item {\tt getSpan(\$time)} 
\item {\tt addLabel(\$time)} 
\item {\tt integrateLabelsFromAlignment(\$alignment)} 
\item {\tt isSilent(\$time)} checks whether 
\item {\tt removeSilentLabels(\$time)} 
\item {\tt removeEmptyLabels(\$time)} 
\item {\tt crop(\$time)} 
\item {\tt makeContinuous(\$time)} 
\item {\tt diffInIUsUpTo (\$time)} 
\item {\tt diffInIUs(\$time)} 
\item {\tt copyUpTo (\$time)} 
\item {\tt equalsUpTo t(\$time)}
\item {\tt prefixUpTo (\$time)} 
\item {\tt collapse(\$time)} 
\item {\tt getTalkSpurts(\$time)} 

\end{itemize}
%new, consistency checking,
%clone, equals,
%TextGridIO, WavesurferIO, SphinxIO,
%simple getters, setters

\subsubsection{TextGrid.pm}
% generelle Dinge - woraus besteht ein TextGrid
{\tt TextGrid} objects are collections of {\tt Alignments}, represented as a list property. 
An object of this kind also contains the features {\tt xmin} and {\tt xmax}, which represent its time span. 
These values can be set automatically with regard to the overall time spanned by the contained alignments or entered manually - in every case, the boundaries {\tt xmin} and {\tt xmax} of each of the alignments have to be within the overall boundaries. 

% Liste von Funktionen
The following operations can be used concerning a {\tt TextGrid} object: 

\begin{itemize}
\item {\tt new(\$xmin, \$xmax, \$alignments, \$format, \$name)} creates a new {\tt TextGrid} object. 
{\tt \$xmin} and {\tt \$xmax} mark its start-time resp.\ end-time. 
If {\tt \$format} is set to 'lax', these values will be calculated and set automatically. 
{\tt \$alignments} is a reference to a list of alignments. {\tt \$name} will be the internal name for that object.
The operation returns a {\tt TextGrid} object. 
Generally, it should be preferred to use one of {\tt newTextGridFromFile()} or {\tt newEmptyTextGrid()}.

\item {\tt checkAlignments(\$alignments)} checks, 
whether all alignments in the list are correct, i.\,e.\ actual alignments by definition. 
Further, the overall upper and lower time boundaries of the alignment list are calculated. 
The operation returns these two overall time boundaries. If no alignments have been found at all, (0,0) is returned.  

\item {\tt newTextGridFromLines(\textbackslash@lines)} creates a new {\tt TextGrid} object 
from given text data ({\tt \textbackslash@lines} as an array of lines) syntactically forming a TextGrid.

{\bf Notice} that only latin-1 encoded TextGrids in the regular text-format (not "short text") are supported. 
Newer versions of Praat often output utf-16 (big endian) encoded TextGrid files, especially if they contain Umlauts.
These have to be converted (e.\,g.\ using the UNIX ``iconv'' or ``recode'' commands) to latin-1. 
TextGrids are automatically read and written as latin-1, regardless of your environment. Thus, there is no need
to recode TextGrids between using Praat and TELIDA.

The operation returns a call of the {\tt new}-operator and therefore a {\tt TextGrid} object. 
  
\item {\tt newTextGridFromFile(\$filename)} creates a new {\tt TextGrid} object 
from a given file {\tt \$filename}, whose content has to syntactically form a TextGrid. 
The operation returns a {\tt TextGrid} object.


\item {\tt newEmptyTextGrid()} creates a new {\tt TextGrid} object without assigning values to its components. 
The operation returns a call of the {\tt new}-operator and therefore a {\tt TextGrid} object.

%%%%% TODO %%%%%%
\item {\tt newTextGridFromS1HFile(\$lines)}
%%%%%%%%%%%%%%%%%


\item {\tt toTextGridLines()} creates writable TextGrid data from a {\tt TextGrid} object. 
It is asserted, that each alignment has the same time boundaries as the TextGrid itself. 
The operation returns an array of lines representing the TextGrid. 

\item {\tt toTEDXMLLines(\$lines, \$nowrap)} creates writable XML data for TEDview from a {\tt TextGrid} object. 
If {\tt \$nowrap} is set to 'nowrap', the top-level 'dialogue' tag is
omitted. 
As TextGrid is a non-incremental format, no enclosing 'diamond' events are generated for
the enclosed tracks.
The originator attribute is passed on to the label.
The operation returns an array of lines representing the TEDXML data. 

\item {\tt toTED(\$ip, \$port)} sends a {\tt TextGrid} object 
to the TEDview application via port {\tt \$port} and IP address {\tt \$ip}. 
These features have default values (port 2000 and IP 127.0.0.1) and are therefore optional.  

\item {\tt saveToTextGridFile(\$filename)} saves the {\tt TextGrid} object to a file {\tt \$filename}.

\item {\tt getAlignmentNames()} creates a list of all alignment names (i.\,e.\ tier names) found in the {\tt TextGrid} object. 
The operation returns an array of alignment names.

\item {\tt getAlignmentsByName(\$name)} gets all alignments that match a given string {\tt \$name}. 
The operation returns an array of alignment names.

\item {\tt getAlignmentByName(\$name)} gets the first alignment name that matches a given string  {\tt \$name}. 
The operation returns the first alignment name in the list of matching names. If no match could be found, this value is empty. 

\item {\tt hasAlignment(\$name)} checks whether a {\tt TextGrid} object has an alignment with the name {\tt \$name}. 
The operation returns true (1) or false (0).   

\item {\tt addAlignment(\$alignment)} adds the alignment {\tt \$alignment} to a TextGrid object. Duplicate names in the alignment list are not allowed. 

\item {\tt addAlignmentAfter(\$newAlignment, \$name) } adds the alignment {\tt \$newAlignment} to a TextGrid object. It will be inserted after the alignment with the name {\tt \$name}.
 
\item {\tt removeAlignmentByName(\$name)} removes the alignment with the name  {\tt \$name} from the list of alignments. 

\item {\tt replaceAlignmentByName(\$name, \$newAlignment )} replaces the alignment with the name {\tt \$name} with the alignment {\tt \$newAlignment}. 

\end{itemize}


\subsection{Incremental Timed Data}

This subsection covers data structures and methods suitable for incremental timed data. 
Obviously, an understanding for the processing of non-incremental timed data (as discussed in the
previous subsection) is vital also for the processing of incremental timed data.

We do not cover collections of data of different types as we have for non-incremental timed data.
Instead we limit ourselves to the generalization of alignment-information to the incremental case
and to the case where multiple hypotheses about alignments are considered at the same time and 
incrementally as time proceeds.

\subsubsection{SingularAlignmentSequence.pm}

A SingularAlignmentSequence contains many alignments (each with a {\tt time}) 
which record what we think at a given point in time.

\subsubsection{NBestList.pm}

An NBestList contains several alignments which we all believe in at a certain 
point in time (not different points in time as for SingularAlignmentSequences, 
see above), ranked by the strength of our belief.

\subsubsection{NBestAlignmentSequence.pm}

An NBestAlignmentSequence contains many NBestLists, representing the change of 
our (complex) belief states over time.

\subsubsection{IUEdit.pm}

\subsection{Util.pm}

The {\tt Util} module bundles together some (partially trivial) functions that
are used in several parts of the code and could be of use to other scripts. They 
are only vaguely related to the processing of timed data.

\begin{itemize}
\item {\tt Util::min(\$a, \$b) :\ \$} and {\tt Util::max(\$a, \$b) :\ \$}\\
 return the (arithmetically) smaller or larger of the given values
\item {\tt Util::minList(@list) :\ \$}\\
 returns the (arithmetically) smallest value in the given list
\item {\tt Util::parseLine(\$line, \$re) :\ \$}\\
 is internally used for TextGrid parsing. Can also be used more generally
 to match a line of text to the given regular expression and returns 
 the first parenthesized part of the regexp ({\tt \$1}) if it exists.
\item {\tt Util::tokenize(\$line) :\ @}\\
 splits a given line of text into word tokens and returns a list of these word tokens. 
\item {\tt Util::levenshtein(\textbackslash@list1, \textbackslash@list2) :\ \$}\\
 return the levenshtein distance between two lists
\item {\tt Util::runCached(\$question, \$processor, \$command) :\ @ / \$}\\
 return the lines of standard output the external command {\tt \$command} will produce 
 when faced with the standard input {\tt \$question}. The method returns either a list
 of lines or a scalar containing all the lines delimited by new lines (\textbackslash{n})
 depending on the caller's preference.

 As the method name implies, the command is not necessarily being run, but
 the result may also be the cached version of a previous call to the external command.
 Cached results are stored in a file system cache (located in the directory {\tt runCache/} 
 relative to the calling script). Within the cache, files are further sub-divided by the
 name given in {\tt \$processor}. 

 WARNING: As results are cached for future use, it is important to use this method only if
 calls to the external command are known to produce identical output when called
 repeatedly with identical input.
 For example, using it to cache complex parsing calculations that do not change between runs
 is advisable. Running it to cache calls to {\tt date} or worse {\tt head /dev/random} may 
 not return what you expect.
\end{itemize}

\section{File Types}

% TODO: document the supported file types
% TextGrid and and Wavesurfer-Labelfile formats can be covered quickly 
% mostly we have to outline what they do and talk about our limitations in TextGrid support
% then talk about our own formats for NBestLists, plain incremental data and
% nbest-incremental data.

NBest lists and especially NBestAlignmentSequences can become rather large and reading
them from a file then becomes slow. For this reason NBestAlignmentSequence::newNBestSeqFromFile()
automatically writes gzipped Perl Storable files (.pls.gz) for each file accessed and will 
use these files to serve later calls. This improves performance by quite a bit.

% finally we should write or link to a TEDviewXML documentation


\section{CLI Applications}

Some of the modules' functionality is available in two command-line applications, one
for non-incremental data and another for incremental data. Focus is on opening files, 
showing contents and manipulating data. 

\subsection{TGtool.pl}

Reads, writes and manipulates TextGrids and Alignments. Alignments can be edited in an 
external editor and visualized in an open TEDview instance.

TGtool supports the following commands:

\begin{description}
  \item[help] provides help on commands.\\ \emph{USE THIS FOR MOST UP-TO-DATE INFORMATION!}
  \item[exit] exits the program.
  \item[] % quasi new-line
  \item[ls] list the working directory
  \item[cd] change the working directory
  \item[pwd] print the working directory
  \item[] % quasi new-line
  \item[new] create a new and empty TextGrid
  \item[load \textlangle filename\textrangle] load a file
  \item[{save [[\textlangle tier\textrangle] \textlangle filename\textrangle]}] 
        if tier is given, write the tier to file, if no tier
        is given, write textgrid to file, if no filename is given, write to the filename
        last opened
  \item[] % quasi new-line
  \item[{show [\textlangle tiername\textrangle]}] 
        list the given tier or list the tier names if no tiername is given
  \item[part \textlangle tiername\textrangle] \textlangle start\textrangle] \textlangle end\textrangle]]
        list the part of the given tiername from start-time to end-time
  \item[] % quasi new-line
  \item[{add \textlangle filename\textrangle [\textlangle name\textrangle [\textlangle prev\textrangle]]}]
        open a WS tier from filename, call it name and add it to the
        current textgrid after the last tier which matches prev
  \item[rm \textlangle tiername\textrangle] remove the tier from the textgrid
  \item[repl \textlangle tiername\textrangle \textlangle filename\textrangle]
        replace the tier with the alignment read from filename
  \item[] % quasi new-line
  \item[edit \textlangle tiername\textrangle] edit the tier in nedit; when you're done editing,
        save the result and close nedit; the tier will be updated.
  \item[{grep \textlangle regexp\textrangle [\textlangle tiername\textrangle \ldots}]
        search all or specific tiers for labeltext matching the regexp
  \item[] % quasi new-line
  \item[TEDview \textlangle port\textrangle] 
        show TextGrid in TEDview (port defaults to 2000, TEDview's default port)
  \item[{mark [\textlangle tiername\textrangle] 
              [\textlangle start\textrangle \textlangle end\textrangle] 
              \textlangle color\textrangle}]
        mark the labels in the given or all tiers between start and end (or all of times are omitted)
        in the given color; TEDview will now display the corresponding labels in the given color
\end{description}

\subsection{INtool.pl}

Reads and manipulates AlignmentSequences. Sequences can be visualized with the help of
TEDview, a viewer for incremental timed data.

\subsection{TGgrep.pl}

Search in TextGrids (or other formats) using regular expressions for the labels and
(if you want) the tracks.

\end{document}
